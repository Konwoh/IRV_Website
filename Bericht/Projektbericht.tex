\documentclass[usegeometry=true]{scrartcl}
\usepackage[ngerman]{babel}
\usepackage[T1]{fontenc}
\usepackage{lmodern}
\usepackage[utf8]{inputenc}
\usepackage{hyperref}
\usepackage{amssymb}
% Dimensionen bitte nicht ändern. 
\usepackage[left=2cm, right=2cm, top=2cm, bottom=2cm, bindingoffset=1cm, includeheadfoot]{geometry}
%Zeilenabstand bitte nicht ändern
\usepackage[onehalfspacing]{setspace}

\usepackage[backend=biber,style=numeric,]{biblatex}\addbibresource{literatur.bib}

\begin{document}

\subject{Projektbericht zum Modul Information Retrieval und Visualisierung Sommersemester 2022}
\title{Analyse des Kaufverhaltens in Einkaufsläden}
%\subtitle{Untertitel}% optional
\author{Konstantin Wohllaub}% obligatorisch
%\date{10.9.2022}
\maketitle% verwendet die zuvor gemachte Angaben zur Gestaltung eines Titels

\section{Einleitung}
Seit jeher gibt es zwischen Unternehmen in der Wirtschaft einen Kampf um steigende Gewinne und zahlende Kunden.
In Zuge dessen wird von Seiten der Unternehmensleitungen versucht, einen Vorteil gegenüber ihren Konkurrenten zu erreichen.
Dieser Vorteil wird vorallem, durch die in den letzten Jahren sehr beliebt gewordene Datenanalyse erreicht.
Bis vor ein paar Jahren war die Verarbeitung von solch großen Datenmengen undenkbar,
jedoch machte der enorme Anstieg von Speicher- und Rechenkapazität sowie Fortschritte in der Softwareentwicklung dies in den letzten Jahren möglich.\cite[1]{emrouznejad2016big}
Da der einfache Besitz dieser großen Datenmengen den Unternehmensführungen von nicht all zu großen Nutzen ist, müssen die Daten noch aufbereitet, analysiert und interpretiert werden.\cite[1]{Tien2013}
Dabei wird versucht bestimmte Muster oder Verhaltensweisen zu finden, die letztendlich zu einem Wettbewerbsvorteil gegenüber anderen Marktteilnehmer führen können.

\noindent Die Einzelhandels-Branche gehört zu den Branchen, die mit am meisten Daten generiert.
Jeden Tag strömen Millionen von Menschen in die Einkaufsläden, um Produkte des alltäglichen Bedarfs zu besorgen.
Da durch jeden einzelnen Kunden und jedes einzelne gekaufte Produkt wertvolle Daten generiert werden, hat die Einzelhandels-Branche den Vorteil auf eine sehr ausführliche
Datenansammlung zurückgreifen zu können.
Aus diesem Grund hat unter anderem Walmart, eines der größten Unternehmen auf der Welt, im Jahr 2015 angekündigt, die größte Private Data Cloud der Welt
zu erstellen.\cite[5]{marr2016big}
Zusätzlich zu der sehr hohen Datendichte, ist die Einzelhandels-Branche auch eine der kompetetivsten Branchen, da es in kaum einer anderen Branche
so viele gleichwertige Akteure am Markt gibt, die miteinander konkurrieren.\cite{Grewal2010}
Aus diesem Grund sind es vorallem Unternehmen, die Supermärkte betreiben, die auf die Datenanalyse zurückgreifen müssen, um sich von der Konkurrenz abzuheben.
Auf Grundlage dessen versuchen die Supermärkte dann ihr Angebot so zu gestalten, dass der Nutzen der Kunden maximiert wird, sodass diese gewillt sind, diesen Einkaufsladen den anderen vorzuziehen.\cite[6]{marr2016big}
Zuvor müsssen sich die Supermärkte jedoch Fragestellungen überlegen, um überhaupt zu wissen, wonach gesucht werden soll.

\noindent Genau hier an diesem Punkt soll die Projektarbeit anknüpfen und aus der Sicht einer Supermarkt-Leitung, sich möglich ergebende Fragen, zu beantworten.
\begin{itemize}
	\item Sind bestimmte Muster in den Daten zu erkennen, die auf Abhängigkeiten zwischen Kaufattributen deuten ?
	\item Welche Kunden sind am profitabelsten ?
	\item Welche Produkte sind am profitabelsten ?
	\item Gibt es Ausreißer und falls ja, warum ?
\end{itemize}

\subsection{Anwendungshintergrund}
erklären um was für ein Datensatz sich es handelt
erklären, dass die Fragestellungen mithilfe von Visualisierung gelöst weren sollen

\subsection{Zielgruppen}
\subsection{Überblick und Beiträge}

\section{Daten}

\printbibliography


\end{document}