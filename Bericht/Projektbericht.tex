\documentclass[usegeometry=true]{scrartcl}
\usepackage[ngerman]{babel}
\usepackage[T1]{fontenc}
\usepackage{lmodern}
\usepackage[utf8]{inputenc}
\usepackage{hyperref}
\usepackage{amssymb}
% Dimensionen bitte nicht ändern. 
\usepackage[left=2cm, right=2cm, top=2cm, bottom=2cm, bindingoffset=1cm, includeheadfoot]{geometry}
%Zeilenabstand bitte nicht ändern
\usepackage[onehalfspacing]{setspace}
\usepackage{graphicx}
\usepackage{float}
\usepackage[backend=biber,style=numeric,]{biblatex}\addbibresource{literatur.bib}

\begin{document}

\subject{Projektbericht zum Modul Information Retrieval und Visualisierung Sommersemester 2022}
\title{Analyse des Kaufverhaltens in Einkaufsläden}
%\subtitle{Untertitel}% optional
\author{Konstantin Wohllaub}% obligatorisch
%\date{10.9.2022}
\maketitle% verwendet die zuvor gemachte Angaben zur Gestaltung eines Titels

\section{Einleitung}
Seit jeher gibt es zwischen Unternehmen in der Wirtschaft einen Kampf um steigende Gewinne und zahlende Kunden.
In Zuge dessen wird von Seiten der Unternehmensleitungen versucht, einen Vorteil gegenüber ihren Konkurrenten zu erreichen.
Dieser Vorteil wird vorallem, durch die in den letzten Jahren sehr beliebt gewordene Datenanalyse erreicht.
Bis vor ein paar Jahren war die Verarbeitung von solch großen Datenmengen undenkbar,
jedoch machte der enorme Anstieg von Speicher- und Rechenkapazität sowie Fortschritte in der Softwareentwicklung dies in den letzten Jahren möglich.\cite[1]{emrouznejad2016big}
Da der einfache Besitz dieser großen Datenmengen den Unternehmensführungen von nicht all zu großen Nutzen ist, müssen die Daten noch aufbereitet, analysiert und interpretiert werden.\cite[1]{Tien2013}
Dabei wird versucht bestimmte Muster oder Verhaltensweisen zu finden, die letztendlich zu einem Wettbewerbsvorteil gegenüber anderen Marktteilnehmer führen können.

\noindent Die Einzelhandels-Branche gehört zu den Branchen, die mit am meisten Daten generiert.
Jeden Tag strömen Millionen von Menschen in die Einkaufsläden, um Produkte des alltäglichen Bedarfs zu besorgen.
Da durch jeden einzelnen Kunden und jedes einzelne gekaufte Produkt wertvolle Daten generiert werden, hat die Einzelhandels-Branche den Vorteil auf eine sehr ausführliche
Datenansammlung zurückgreifen zu können.
Aus diesem Grund hat unter anderem Walmart, eines der größten Unternehmen auf der Welt, im Jahr 2015 angekündigt, die größte Private Data Cloud der Welt
zu erstellen.\cite[5]{marr2016big}
Zusätzlich zu der sehr hohen Datendichte, ist die Einzelhandels-Branche auch eine der kompetetivsten Branchen, da es in kaum einer anderen Branche
so viele gleichwertige Akteure am Markt gibt, die miteinander konkurrieren.\cite{Grewal2010}
Aus diesem Grund sind es vorallem Unternehmen, die Supermärkte betreiben, die auf die Datenanalyse zurückgreifen müssen, um sich von der Konkurrenz abzuheben.
Auf Grundlage dessen versuchen die Supermärkte dann ihr Angebot so zu gestalten, dass der Nutzen der Kunden maximiert wird, sodass diese gewillt sind, diesen Einkaufsladen den anderen vorzuziehen.\cite[6]{marr2016big}
Zuvor müsssen sich die Supermärkte jedoch Fragestellungen überlegen, um überhaupt zu wissen, wonach gesucht werden soll.

\noindent Genau hier an diesem Punkt soll die Projektarbeit anknüpfen und aus der Sicht einer Supermarktleitung, sich möglich ergebende Fragen, zu beantworten.

\begin{itemize}
	\item Sind bestimmte Muster in den Daten zu erkennen, die auf Abhängigkeiten zwischen Kaufattributen deuten ?
	\item Welche Kunden sind am profitabelsten ?
	\item Welche Produkte sind am profitabelsten ?
	\item Gibt es Ausreißer und falls ja, warum ?
\end{itemize}

\subsection{Anwendungshintergrund}
Bei dem Kauf eines Produkt in einem Supermarkt werden, wie in der Einleitung bereits beschrieben, viele Informationen generiert. Unteranderem kann festgestellt werden,
welches Geschlecht der Kunde hat, was er für Produkte einkauft, welchem Segment diese Produkte zugehörig sind, wie viele Produkte von einer Sorte gekauft werden,
wie viel er dafür insgesamt bezahlt und vieles mehr. Im optimalen Fall sammeln Supermärkte diese Daten, um sie in einem Datensatz zu speichern.
Der Datensatz der im Zuge dieser Arbeit analysiert und visualisiert wurde, ist das Ergebnis einer solchen Datensammlung. Nun kann durch spezielle
Informationsvisualisierungstools dieser Datensatz ausgewertet werden, um Rückschlüsse oder Vorhersagen bzgl. des Kundenverhaltens machen zu können.
Da es aber oftmals schwer ist Muster in der Gesamhtheit der Kundendaten zu erkennen, kann es, speziell für Supermärkte, sinnvoll sein die Daten nach gewissen Eigenschaften zu
filtern. Dies ermöglicht es zum Beispiel Information bzgl. des Kaufverhaltens von Männern und Frauen sowie von einkommenstarken und einkommenschwachen Kunden zu bekommen.


\subsection{Zielgruppen}
Die Zielgruppe dieser Arbeit besteht in erster Linie aus den Unternehmensführungen von Supermärkten, die sich in Ländern mit ähnlichen demographischen und kuluterellen
Gegebenheiten befinden, wie die Supermärkte von denen die Daten gesammelt wurden. Die Unternehmensleitungen können aus der Arbeit wichtige Informationen hinsichtlich ihres
operativen Geschäfts ziehen undihr bereits vorhandenes Wissen damit erweitern. Zudem können die Ergebnisse dieser Arbeit auch für die Marketing-Abteilungen der Unternehmen
interessant sein, da sie durch die zusätzlich Informationen über ihre Kunden, die Werbekampagnen besser auf die Bedürfnisse der Kunden abstimmen können.

\noindent Auf der anderen Seite können auch die Kunden selbst, von den Visualisierungen dieser Arbeit profitieren. Dadruch, dass nicht nur das Kundenverhalten im Fokus steht,
sondern auch Attribute wie z.B. Kundenzufriedenheit oder Verkaufspreis, können die Kunden einsehen, wo der Service und die Preise am optimalsten für sie selbst sind.

\subsection{Überblick und Beiträge}
In dem verwendeten Datensatz sind alle Einkäufe aufgelistet, die es in einem Zeitraum von drei Monaten in den drei myanmarnesischen Städten Yangon, Naypyitaw, Mandalay gab.
Dabei wurden die drei Zweigstellen in den drei Städten, der nicht namentlich genannten Supermarktkette, betrachtet. In dem Datensatz wurden produkt- und kundenspezifische
Daten quantitativer und qualitativer Natur erfasst. Bei den produktspezifischen Daten handelt es sich nicht um einen kompletten Warenkorb, sondern
um einzelne Produkte und deren gekauften Anzahl.

\noindent Die Umsetzung der Visualisierung erfolgt durch genau drei Techniken: Scatterplot, Parallele Koordinaten und eine Pixelorientierte Darstellung mithilfe der Recursive Pattern
Technik. Auf die einzelnen Visualisierungstechniken wird im späteren Verlauf der Arbeit noch genauer eingegangen.

\noindent Der Beitrag dieser Arbeit setzt sich aus der Visualisierung des zugrunde gelegten Datensatzes und denen daraus resultierenden Informationen zusammen. Der Lesende erhält
dadruch ein Einblick in das Kunden- und Kaufverhalten im Bezug auf Supermärkte. Diesen Einblick und die damit erlangten Informationen können dann genutzt werden, um Sie für
weiterführende Forschungszwecke ökonomischer Natur einzusetzen.

\section{Daten}
Wie oben bereits angedeutet, ist der in der Arbeit verwendete Datensatz eine Sammlung von Kunden- und Kaufdaten aus drei verschiedenen Zweigstellen eines Supermarkt
Unternehmens in Myanmar. Die verschiedenen Zweigstellen befinden sich in den Städten Yangon, Naypyitaw und Mandalay. Mandalay und Naypyitaw haben knapp über 1 Million Einwohner,
Yangon sogar über 5 Millionen Einwohner. Jede Zeile des Datensatzes besteht aus einer Computergenerierten Rechnungsnummer, zur eindeutigen Identifikation des Kaufes, sowie dem
Produkt, dass gekauft wurde. Hinzu kommen Daten über den Kunden, wie Geschlecht, Kundenstatus, Einkommen des Kunden und Bewertung des Kaufes. Über das Produkt gibt es folgende
Informationen: Verkaufspreis, Produktlinie, Kosten des Produkts für den Supermarkt, Anzahl gekaufter Produkte, Steuer, Preis inklusive Steuer und die prozentuale Gewinnspanne.
Dieser Umfang an Daten je getätigten Kauf stellt sicher, dass eine umfassende Analyse, bezüglich des Kaufverhaltens der Kunden sowie der Verkaufsperformance des Supermarkts,
möglich ist. Alle in der Einleitung aufgestellten Fragen können somit beantwortet werden. Durch Daten auf Kunden- und Produktseite ist es zudem möglich die verschiedenen
Zielgruppen, welche bereits definiert wurden, mit den geeigneten Informationen zu bedienen. Aus diesem Grund war es nicht nötig, den Datensatz mit zusätzlichen Daten zu erweitern.

\subsection{Technische Bereitstellung der Daten}
Der Datensatz ist auf der Website von kaggle unter den Link: \url{https://bit.ly/3eco5mX} zu finden. Dieser ist dort als csv-Datei frei verfügbar. Die csv-Datei wurde im
folgenden repository: \url{https://github.com/Konwoh/Information-Retrieveal} auf GitHub gespeichert. Anschließend wurde die raw-Version der csv-Datei durch eine HTTP-Anfrage
in das Programm geladen, um dort weiterverarbeitet zu werden. Der Datensatz wurde auch zusätzlich in dem finalen repository auf GitLab zur Überprüfung bereitgestellt.

\subsection{Datenvorverarbeitung}
Da der Datensatz in einer einzigen csv-Datei bereits vorlag, ersparte dies etliche Schritte der Datenvorverarbeitung. Einzig allein die Datumsspalte wurde vor dem Einladen
in das Programm mithilfe der Python Bibliothek Pandas verändert. Dabei wurde das Datum, dass in der csv-Datei als String in der Form YYYY/MM/DD gespeichert war, in ein
Datumobjekt der Form YYYY-MM-DD geändert. Dieser Schritt war notwendig, da das ELM-Paket Date (\url{https://package.elm-lang.org/packages/justinmimbs/date/latest/Date}),
nur Datumsangaben in der Form YYYY-MM-DD akzeptiert. Der beschriebene Schritt war auch gleichzeitig der einzige Schritt der nicht in ELM programmiert wurde.

\noindent Da die csv-Datei nun im ELM-Programm zur Verfügung steht, mussten nun Datentypen definiert werden, in denen die Zeileninhalte des Datensatzes überschrieben werden.
Hierfür gab es den primären Datentyp Sale und mehrere Unterdatentypen wie z.B. das Geschlecht oder die Produktlinie. Nichtsdestotrotz waren die Unterdatentypen trotzdem ein
Bestandteil des primären Datentyps Sale. Damit die Überschreibung der Daten von der csv-Datei in die gewünschten Datentypen erfolgreich stattfindet, musste ein Decoder
geschrieben werden. Die salesDecoder-Funktion, die Funktionalitäten des Elm-Pakets elm-csv (\url{https://package.elm-lang.org/packages/BrianHicks/elm-csv/latest/}) benutzt, war
dafür verantwortlich. Mit Anwenden dieser Funktion auf den geladenen Datensatz, war die Datenvorverarbeitung nun abgeschlossen.

\section{Visualisierungen}
\subsection{Analyse der Anwendungsaufgaben}
Die Anwendungsaufgaben der Anwender und Anwenderinnen ergeben sich zunächst aus der bloßen Betrachtung und den damit verbundenen Verstehen der einzelnen Visualisierungstechniken.
Danach sollen durch die verfügbaren Variationen, die von den Visualisierungstools zur Verfügung gestellt werden, genau die Attribute betrachtet werden, die für den Anwender
interessant und wichtig erscheinen. Dadruch sollte es den Betrachter möglich sein, die in der Einleitung aufgestellten Fragestellungen zu beantworten. Dabei ist auch wichitg zu
erwähnen, dass die betrachtete Stichprobe von den Betrachter eingekrenzt werden kann. Dies kann z.B. nach Geschlecht oder Art der Zahlung erfolgen, was sich ebenfalls positiv
auf die Beanwortung der Fragestellungen auswirken sollte.

\noindent Die Visualisierungen zeigen den Betrachtern bestimmte Zusammenhänge zwischen den gewählten Attributen auf, was unter anderem Rückschlüsse auf Korrelation
und Beziehung der einzelnen Attribute untereinander ermöglicht. Da durch die Visualisierung die Daten in einen bestimmten Kontext gerückt werden, wird es dem Betrachter somit
ermöglicht, die Daten besser in der realen Welt einzuordnen. Die daraus resultierenden und besser zu greifenden Vorstellungen können den Anwender schließlich bei der Lösung der
Aufgaben ebenfalls behilflich sein.

\noindent Die eben beschriebenen Vorstellungen oder mentalen Modelle sind sehr wichtig für die Beantwortung der Aufgaben, da z.B. auf die Frage wer der profitabelste Kunde sei,
keine einfache Antwort zu geben sei ohne alle Zusammenhänge sowie Beziehungen der Attribute zueinander graphisch darzustellen. So gibt es zum Beispiel Kunden, die zwar viele
Produkt kaufen, aber dafür nur die billligsten. Andererseits gibt es Kunden, die nur wenige Produkte kaufen, die dafür aber sehr teuer sind. Mentale Modelle lassen sich jedoch
durch diese textlich beschriebenen Vebrindungen und Abhängigkeiten nur sehr schwer bilden, weshalb es umso wichtiger ist, dies in der visuellen Form dazustellen.

\subsection{Anforderungen an die Visualisierungen}
Laut Schumann und Müller sollen Visualisierungen expressiv, effektiv und angemessen sein. \cite[9]{schumann2013visualisierung} Genau diese grundlegenden Anforderungen bestehen
auch bei der Visualisierung der hier verwendeten Daten.

\noindent Um die zu Beginn aufgestellten Fragestellungen zu beantworten ist auch die Existenz von spezifischeren Anforderungen notwendig. Als erstes wäre da eine Filterung der
Daten nach den gewünschten Kriterien zu nennen. Denn um eine Antwort auf die Frage: "Wer sind die profitabelsten Kunden?" zu finden, muss erstmal eine Filterung nach Kundengruppen
möglich sein. Das gleiche gilt auch für die Frage bezüglich der Produkte und der anschließenden FIlterung nach Produktgruppen. Die Filterung nach den gewünschten Kriterien
sollte der Anwender dann auch selbst durch Buttons oder Textfelder vollziehen können.

\noindent Desweiteren muss eine Visualisierungstechnik auch ermöglichen, Zusammenhänge zwischen Attributen in den Kontext des gesamten Datensatzes zu setzen. Wenn in einer Frage,
die Profitabilität behandelt wird, können z.B. mehrere Attribute für die Profitabilität stehen und nicht nur der Verkaufspreis eines Produkts, sondern auch die
Anzahl, wie oft es gekauft wurde oder die Kosten, die der Supermarkt für dieses Produkt zahlen musste. Durch den Einfluss mehrere Attribute auf die Beantwortung einer Frage, muss
es daher möglich sein, die Daten mehrdimensional darzustellen.

\noindent Als letzte Anforderung wäre noch die Identifikation von Ausreißern in den Daten zu nennen. Hier sollte es möglich sein, Daten, die sich deutlich von der Norm abheben,
auszumachen und Details zu ihnen einsehen können.

\subsection{Präsentation der Visualisierungen}
\subsubsection{Visualisierung 1: Scatterplot}
Die erste gewählte Visualisierungstechnik ist ein Scatterplot oder auch Streudiagramm. Bei einem Scatterplot werden zwei numerische Attribute mittels zwei Achsen skaliert und die
jeweiligen Datenwerte einem x- sowie y-Wert zugewiesen. Diese x- und y-Werte werden dann in einem Punkt zusammengefasst und in dem Scatterplot gezeichnet. Sinn eines Scatterplot
ist es Zusammenhänge oder andere auffällige Muster zwischen numerischen Attributen festzustellen. \cite[103]{Friendly2005} Welches numerische Attribut auf welcher Achse
abgebildet wird ist im Zuge dieses Projekts vom Anwender frei wählbar. Zudem werden nur die Datenpunkte dargestellt, die zuvor vom Anwender nach gewünschten nominalen
Attributen herausgefiltert wurden.
\begin{figure} [H]
	\begin{center}
		\includegraphics[width=15cm]{IMG/Scatterplot}
		\caption{Visualisierung Eins - Scatterplot}
		\label{fig:Scatterplot}
	\end{center}
\end{figure}

\noindent Der Scatterplot erfüllt sehr gut die grundlegenden Anforderungen der Expressivität, der Effektivität und der Angemessenheit. Der Scatterplot ist expressiv, weil er
sehr intuitiv zu verstehen ist und deutlich zeigt, ob Zusammenhänge oder sonstige Muster zwischen den zwei gewählten Attributen vorherrschen. Zudem ist die eindeutige
Zuweisung der Datenpunkte zu ihren jeweiligen Reihen in dem Datensatz durch das Anzeigen einer eindeutigen Identifikationnummer, während des Hovern über dem
Datenpunkt, gewährleistet. Der Scatterplot ist effektiv, weil er ohne großen Aufwand sehr viele und nützliche Informationen darstellen kann. Und er ist angemessen, da
es eine der trivialsten Möglichkeiten ist, um zwei numerische Attribute aufeinander darzustellen. Daten-Ausreißer sind bei dieser Visualisierungstechnik ebenfalls leicht zu
identifizieren, da die Entfernung von Punktelinien oder Punktewolken sehr gut erkennbar ist.
werden.\\

\noindent Die Scatterplot-Technik ist für den zu untersuchenden Datensatz perfekt geeignet, weil für die Beantwortung einiger Fragestellungen immer eine Betrachtung von
zwei Variablen in verschiedenen Kombinationen notwendig ist. Damit es möglich z.B die Beziehung von der Variable \textit{Anzahl gekaufter Produkt} mit der Variable
\textit{Bruttoeinkommen} oder die Beziehung der Variablen \textit{Rating} und \textit{Gesamtpreis} zu vergleichen. Hierbei kann der Anwender nach belieben die ihn interessierenden
Attribute miteinander kombinieren.
Des Weiteren kann durch die eindeutige Zuweisung der Datenpunkte zu ihren Datenreihen, zusätzliche Analysen bzgl.
des gewählten Datenpunktes stattfinden. Diese Analysen könnten im weitern Verlauf dazu genutzt werden, um an weitere Erkenntnisse zu gelangen. 
\noindent Ebenbürtige Alternativen zum Scatterplot sind kaum verfügbar. So sagten unteranderem schon Friendly und Denis in ihrem Werk über Scatterplots:
\begin{quote}"Indeed, among all the forms of statistical
	graphics, the humble scatterplot may be considered
	the most versatile, polymorphic, and generally useful invention in the entire history of statistical graphics."\cite[103]{Friendly2005}
\end{quote}

Diese Aussage unterstreicht nochmal die Bedeutung die Scatterplots in der statistischen Analyse besitzen. 
Die einzige Alternative, die in Erwägung gezogen werden kann, ist der Quantil-Quantil-Plot. Hierbei werden ebenfalls zwei numerische Attribute auf Achsen skaliert und 
miteinander verglichen, allerdings beschränkt sich hier der Vergleich auf die Verteilungen der beiden Variablen. Aus diesem Grund ist diese Visualisierungstechnik nicht wirklich
für die Anwendung im Rahmen dieses Projekts geeignet, da die Verteilungen der Variablen für die Beantwortung der Problemstellungen keine Rolle spielen.

\printbibliography


\end{document}